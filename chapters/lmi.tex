\chapter{Γραμμικές ανισότητες πινάκων}\label{ch:lmi}
Τα τελευταία χρόνια, οι γραμμικές ανισότητες πινάκων (\tl{LMI}) έχουν αναδυθεί
ως ένα ισχυρό εργαλείο προσέγγισης προβλημάτων ελέγχου που φαίνονται δύσκολα, αν
όχι αδύνατον να επιλυθούν αναλυτικά. Η ιστορία των γραμμικών ανισοτήτων πινάκων
ξεκινά γύρω στο \(1890\) με τον \tl{Lyapunov} και τη μελέτη ευστάθειας δυναμικών
συστημάτων. Τη δεκαετία του σαράντα, επιστήμονες από τη Σοβιετική Ένωση
εφάρμοσαν τις ιδέες του \tl{Lyapunov} προσπαθώντας να λύσουν προβλήματα ελέγχου.
Αν και δεν διατύπωσαν τη μορφή των γραμμικών ανισοτήτων πινάκων, η συνθήκη
ευστάθειας είχε τη μορφή \tl{LMI}. Μεγάλη έξαρση στη θεωρία τους
παρουσιάστηκε τη δεκαετία του εξήντα με τη δουλεία των \tl{Kalman, Yakubovich,
Popov} και άλλων, με ιδιαίτερη έμφαση στις εφαρμογές των \tl{LMI} στη θεωρία
ελέγχου. Τις τελευταίες δεκαετίες αναπτύχθηκαν ισχυροί αλγόριθμοι για την
αριθμητική επίλυση των γραμμικών ανισοτήτων πινάκων. Η ιδέα των \tl{LMI} και των
εφαρμογών τους βασίζονται στο γεγονός ότι τα \tl{LMI} μπορούν να
μετασχηματιστούν σε ισοδύναμα προβλήματα κυρτού προγραμματισμού τα οποία μπορούν
εύκολα να επιλυθούν.

Στο κεφάλαιο αυτό θα παρουσιάσουμε κάποια βασικά θεωρητικά στοιχεία των
γραμμικών ανισοτήτων πινάκων με βάση τα βιβλία \cite{boyd1994linear} και
\cite{scherer2011linear}. Στη συνέχεια θα παρουσιάσουμε κάποια
παραδείγματα \tl{LMI} στο \tl{MATLAB} χρησιμοποιώντας το \tl{Robust Control
Toolbox}.

\section{Θεωρητικά στοιχεία}
Μία γραμμική ανισότητα πινάκων είναι της μορφής
\begin{equation}\label{eq:lmi}
    F(x) = F_0 + \sum_{i=1}^m x_iF_i > 0,
\end{equation}
όπου \(x = (x_1 \dots x_m) \in \mathbb{R}^m\) είναι η μεταβλητή βελτίστου και οι
συμμετρικοί πίνακες \(F_i = F_i^T \in \mathbb{R}^{n \times n} \) για \(i = 0,
\dots, m \) δίνονται. Το σύμβολο ανισότητας στην παραπάνω σχέση σημαίνει ότι η
\(F(x)\) είναι θετικά ορισμένη. Πολλές φορές συναντάμε \tl{LMI} όπου η \(F(x)
\geq 0\) είναι θετικά ημι-ορισμένη.

Το \tl{LMI} \eqref{eq:lmi} είναι ένας κυρτός περιορισμός στο \(x\), δηλαδή το σύνολο
\( \{ x | F(x) > 0 \} \) είναι κυρτό. Αν και φαίνεται ότι \tl{LMI} έχει
ειδική μορφή, μπορεί να συμβολίσει μεγάλο εύρος κυρτών περιορισμών που
παρουσιάζονται σε φυσικά προβλήματα. Πιο συγκεκριμένα, γραμμικές ανισότητες,
(κυρτές) τετραγωνικές ανισότητες, περιορισμοί που εμφανίζονται σε προβλήματα
της θεωρίας ελέγχου, όπως συναρτήσεις \tl{Lyapunov} και κυρτές τετραγωνικές
ανισότητες πινάκων.

Ένα σύστημα \tl{LMI} \( F^{(1)}(x) > 0, \dots, F^{(p)}(x) > 0 \) μπορεί να
μετασχηματιστεί στην ισοδύναμη μορφή ενός \tl{LMI}
\begin{equation*}
    \begin{pmatrix}
        F^{(1)}(x) & {} & {} & {}\\
        {} & F^{(2)}(x) & {} & {}\\
        {} & {} & \ddots & {} \\
        {} & {} & {} & F^{(p)}(x)
    \end{pmatrix} > 0.
\end{equation*}

Ιδιαίτερα χρήσιμο είναι το συμπλήρωμα του \tl{Schur} που μας επιτρέπει να
μετατρέψουμε μη-γραμμικές ανισότητες σε \tl{LMI}. Πιο συγκεκριμένα,
\begin{equation*}
    F(x) =
    \begin{pmatrix}
        F_{11}(x) & F_{12}(x) \\
        F_{21}(x) & F_{22}(x)
    \end{pmatrix}
\end{equation*}
όπου \(F_{11}(x)\) είναι τετράγωνος. Τότε \(F(x) > 0\) αν και μόνο αν
\begin{align*}
    &F_{22}(x) > 0 \\
    &F_{11}(x) - F_{12}(x)F_{22}^{-1}(x)F_{21}(x) > 0,
\end{align*}
ή ισοδύναμα
\begin{align*}
    &F_{11}(x) > 0 \\
    &F_{22}(x) - F_{21}(x)F_{11}^{-1}(x)F_{12}(x) > 0.
\end{align*}
Έτσι αν κάθε \(F_i(x)\) είναι γραμμική συνάρτηση τότε μπορούμε να μετατρέψουμε
τη μη-γραμμική ανισότητα σε \tl{LMI}.

\section{Κατηγορίες γραμμικών ανισοτήτων πινάκων}
Πολλά προβλήματα βελτιστοποίησης στον έλεγχο, στην επεξεργασία σήματος και τα
λοιπά, μπορούν να διατυπωθούν ως γραμμικές ανισότητες πινάκων. Προφανώς, έχει
νόημα να εκφράσεις τα προβλήματα αυτά ως \tl{LMI} αν μπορούν να επιλυθούν με
αποδοτικό και αξιόπιστο τρόπο. Κατά τη μελέτη γραμμικών ανισοτήτων πινάκων
διακρίνουμε δύο κατηγορίες προβλημάτων:
\begin{enumerate}
    \item \textbf{Εφικτότητα}: Το ερώτημα αν υπάρχει \(x\) τέτοιο ώστε \(F(x) >
        0 \) ονομάζεται πρόβλημα εφικτότητας. Το \tl{LMI} \(F(x) > 0 \)
        καλείται εφικτό αν υπάρχει τέτοιο \(x\), αλλιώς καλείται μη-εφικτό.
    \item \textbf{Βελτιστοποίηση}: Έστω η αντικειμενική συνάρτηση \(f\) που
        υπόκειται σε περιορισμούς της μορφής \( \{ x | F(x) > 0 \} \). Το
        πρόβλημα έγκειται στην εύρεση του \(x\) το οποίο θα ελαχιστοποιεί την
        αντικειμενική συνάρτηση.
\end{enumerate}
Στη συνέχεια θα δώσουμε κάποια τυπικά απλά παραδείγματα των παραπάνω
κατηγοριών.

\subsection{Παράδειγμα 1: ευστάθεια}
Θα εξετάσουμε την εκθετική ευστάθεια του γραμμικού αυτόνομου συστήματος
\begin{equation*}
    \dot{x} = A x
\end{equation*}
όπου \( A \in \mathbb{R}^{n \times n} \). Με αυτό εννοούμε αν υπάρχει ή δεν
υπάρχει θετικός και σταθερός \(M\) και \( a > 0 \) τέτοιοι ώστε για κάθε αρχική
συνθήκη \(x_0\) η λύση \(x(t)\) με \(x(t_0) = x_0 \) να ικανοποιεί το όριο
\begin{equation*}
    \|x(t)\| \leq \|x(t_0)\|Me^{-a(t-t_0)},\quad \text{\gr{για κάθε }} t \geq
    t_0.
\end{equation*}
Από τον \tl{Lyapunov} γνωρίζουμε ότι το σύστημα είναι εκθετικά ευσταθές αν και
μόνο αν υπάρχει \( X = X^T \) τέτοιο ώστε \(X > 0\) και \( A^TX + XA < 0 \).
Συνεπώς, η εκθετική ευστάθεια του συστήματος είναι ισοδύναμο με το πρόβλημα
εφικτότητας \tl{LMI} της μορφής
\begin{equation*}
    \begin{pmatrix}
        -X & 0 \\
        0 & A^TX + XA
    \end{pmatrix} < 0.
\end{equation*}

\subsection{Παράδειγμα 2: εκτίμηση τετραγωνικού κόστους}
Έστω το γραμμικό αυτόνομο σύστημα
\begin{equation*}
    \dot{x} = Ax, \quad x(0) = x_0
\end{equation*}
μαζί με το δείκτη επίδοσης \( J = \int_0^{\infty} x^T(t) Q x(t) \, dt \) όπου \(
Q = Q^T \geq 0 \). Έστω ότι το σύστημα είναι ασυμπτωτικά ευσταθές. Τότε όλες οι
λύσεις του \(x\) είναι τετραγωνικά ολοκληρώσιμες ώστε \( J < \infty \).
Εξετάζοντας τη μη-αυστηρή γραμμική ανισότητα πινάκων \(X \geq 0 \) και
\( A^TX XA + Q \leq 0 \). Για κάθε εφικτή λύση \( X = X^T \) μπορούμε να
παραγωγίσουμε την ποσότητα \( x^T(t)Xx(t) \) ως προς το χρόνο
\begin{equation*}
    \frac{d}{dt}\left[ x^T(t)Xx(t) \right] = x^T(t)\left[A^TX + XA \right] x(t)
    \leq - x^T(t)Qx(t).
\end{equation*}
Ολοκληρώνοντας την τελευταία από \(t = 0 \) έως \(\infty\) παίρνουμε το πάνω
όριο
\begin{equation*}
    J = \sum_0^{\infty}x^T(t)Qx(t) \, dt \leq x^T_0Xx_0,
\end{equation*}
όπου χρησιμοποιήσαμε \(\lim_{t \to \infty} x(t) = 0 \). Επίσης, το μικρότερο
πάνω όριο της \(J\) προκύπτει ελαχιστοποιώντας τη συνάρτηση \(f(X) = x^T_0Xx_0
\) ως προς το \(X=X^T\) που ικανοποιεί τις ανισώσεις \( X \geq 0 \) και
\(A^T + XA + Q \leq 0 \). Τελικά, προκύπτει ένα πρόβλημα ελαχιστοποίησης με
\tl{LMI} περιορισμούς.

\section{Επίλυση}
Προβλήματα \tl{LMI} μπορούν να λυθούν στο \tl{MATLAB} με το \tl{Robust Control
Toolbox}. Η διαδικασία για να ορίσουμε το πρόβλημα είναι αρκετά περίπλοκη, για
αυτό θα περιγράψουμε τα βήματα που απαιτούνται.
\begin{enumerate}
    \item \textbf{Αρχικοποίηση του \tl{LMI} συστήματος} Ένα σύστημα \tl{LMI}
        αρχικοποιείται με τη συνάρτηση \tl{\texttt{setlmis([])}}. Σε περίπτωση
        που θέλουμε να επεξεργαστούμε ένα υπάρχον \tl{LMI} αυτό γίνεται
        δηλώνοντας το όνομα μέσα στις παρενθέσεις. Για παράδειγμα εάν το υπάρχον
        \tl{LMI} ονομάζεται \tl{\texttt{lmi0}} τότε με τη \tl{\texttt{setlmis(lmi0)}}
        μπορούμε να το επεξεργαστούμε.
    \item \textbf{Δήλωση μεταβλητών βελτιστοποίησης} Η δήλωση των μεταβλητών
        βελτιστοποίησης γίνεται με τη συνάρτηση
        \tl{\texttt{X = lmivar(type, struct)}} ορίζουμε τη μεταβλητή πίνακα που
        θέλουμε να βελτιστοποιήσουμε για το \tl{LMI} σύστημα που μελετάμε. Με
        την πρώτη είσοδο της συνάρτησης, \tl{\texttt{type}}, καθορίζουμε τον
        τύπο της μεταβλητής και με το \tl{\texttt{struct}} δίνουμε επιπλέον
        στοιχεία για τη δομή του πίνακα.
        \begin{enumerate}
            \item \textbf{\tl{type=1}} Τότε έχουμε
                συμμετρικούς πίνακες με μπλοκ διαγώνια μορφή. Κάθε διαγώνιο μπλοκ είναι
                είτε φουλ (συμμετρικός πίνακας), είτε βαθμωτό μέγεθος (πολλαπλάσιο
                μοναδιαίου πίνακα), είτε μηδενικός πίνακας. Αν \mono{X} έχει \mono{R}
                διαγώνια μπλοκ, τότε η δομή \mono{struct} είναι \(R \times 2\) πίνακας
                όπου, \mono{struct(r, 1)} είναι το μέγεθος του \(r-\)μπλοκ και
                \mono{struct(r, 2)} είναι ο τύπος το \(r-\)μπλοκ ( \(1\) για φουλ, \(0\)
                για βαθμωτό και \(-1\) για μηδενικό μπλοκ).
            \item \textbf{\tl{type=2}} Για \( m \times n\) ορθογώνιο πίνακα. Το
                \mono{struct = [m, n]} στην περίπτωση αυτή.
            \item \textbf{\tl{type=3}} Για άλλες μορφές πινάκων. Με την τιμή
                αυτή κάθε στοιχείο του \mono{X} καθορίζεται μηδενικό ή \( \pm
                x_n\) όπου \(x_n\) δηλώνει την \(n-\)μεταβλητή βελτιστοποίησης.
        \end{enumerate}
    \item \textbf{Δήλωση των όρων του \tl{LMI}} Κάθε όρος του \tl{LMI} δηλώνεται
        ξεχωριστά, με τη συνάρτηση \mono{lmiterm(termID, A, B, flag)}.
        Οι όροι του \tl{LMI} διαχωρίζονται σε εξωτερικούς όρους, σε σταθερούς
        όρους και μεταβλητούς όρους. Όταν δηλώνουμε τους όρους του \tl{LMI}
        μπορούμε να δηλώσουμε μόνο τους όρους πάνω ή κάτω από τη διαγώνιο,
        το πρόγραμμα θα δηλώσει αυτόματα τον όρο που παραλείπουμε. Ο πρώτος
        όρος του \mono{termID} μπορεί να είναι
        \begin{equation*}
            \mono{termID(1)}=
            \begin{cases}
                +p \\
                -p
            \end{cases}
        \end{equation*}
        όπου \(+p\) δηλώνει τους όρους αριστερού μέλους του \tl{LMI} και \(-p\)
        τους δεξιούς όρους. Επίσης, η τιμή του \( \| p\| \) έχει να κάνει με το
        \tl{LMI} που περιγράφουμε. Όπως είπαμε, μπορούμε να έχουμε πολλά
        \tl{LMI} και έτσι η τιμή του \( p \) αντιστοιχείται από τη συνάρτηση
        \mono{newlmi}.

        Οι υπόλοιπες τιμές του \mono{termID} είναι
        \begin{equation*}
            \mono{termID(2:3)}=
            \begin{cases}
                [0, 0] \quad \text{\gr{για τους εξωτερικούς όρους}},\\
                [i, j] \quad \text{\gr{για τη θέση του μπλοκ του εσωτερικού
                όρου}},
            \end{cases}
        \end{equation*}
        και
        \begin{equation*}
            \mono{termID(4)}=
            \begin{cases}
                0 \quad \text{\gr{για τους εξωτερικούς όρους}},\\
                x \quad \text{\gr{για τους όρους }} AXB \\
                -x \quad \text{\gr{για τους όρους }} AX^TB,
            \end{cases}
        \end{equation*}
        όπου \(x\) είναι ότι επιστρέφεται από τη συνάρτηση \mono{lmivar}
        για τη μεταβλητή πίνακα \mono{X}, όπως δηλώθηκε δηλαδή στο προηγούμενο
        βήμα.

        Τα στοιχεία \mono{A, B} είναι τα αριθμητικά δεδομένα του προβλήματος. Αν
        κάποιος πίνακας δεν εμφανίζεται στον όρο του \tl{LMI} που δηλώνουμε τότε
        παίρνει την τιμή \(1\). Αν είναι σταθερός όρος ή εξωτερικός όρος τότε ο
        \mono{B} παραλείπεται.

        Τέλος, αν \mono{flag = 's'} τότε προσθέτεται ο ανάστροφος του όρου στον
        όρο που δηλώσαμε με τη συνάρτηση \mono{lmiterm}. Για παράδειγμα η δήλωση
        \mono{lmiterm([1 1 1 X], A, 1, 's')} μας δίνει τον όρο \(AX + X^TA^T\)
        του \((1, 1)-\)μπλοκ του πρώτου \tl{LMI} και είναι ισοδύναμο με τις εντολές
        \mono{lmiterm([1 1 1 X], A, 1)} και \mono{lmiterm([1 1 1 -X], 1, A')}.
    \item \textbf{Επιβεβαίωση του \tl{LMI}} Εφόσον έχουμε ολοκληρώσει τις
        δηλώσεις του συστήματος των \tl{LMI} που θέλουμε να λύσουμε, με τη
        συνάρτηση \mono{getlmis} επιβεβαιώνεται ότι δεν έχουμε εισάγει σωστά τα
        δεδομένα και τελικά επιστρέφεται από τη συνάρτηση η εσωτερική αναπαράσταση
        του προβλήματος.
    \item \textbf{Επίλυση του \tl{LMI}} Τελευταίο βήμα είναι να λύσουμε το
        σύστημα \tl{LMI}. Το \tl{MATLAB} επιλύει προβλήματα εφικτότητας με τη
        συνάρτηση
        \begin{otherlanguage}{english}
            \begin{center}
                [tmin, xfeas] = feasp(lmisys, options, target)
            \end{center}
        \end{otherlanguage}
        όπου υποχρεωτικό όρισμα της παραπάνω συνάρτησης είναι το \mono{lmisys} που
        θέλουμε να λύσουμε. Μας επιστρέφει το \mono{xfeas} που είναι διάνυσμα
        που σχετίζεται με τις μεταβλητές απόφασης  και όχι με τον πίνακα
        βελτιστοποίησης του \tl{LMI}. Για αυτό απαιτείται η χρήση της συνάρτησης
        \mono{dec2mat}. Επίσης, αν το \mono{tmin} είναι μη θετικό τότε είναι
        εφικτό το πρόβλημα και αν είναι αρνητικό τότε είναι αυστηρώς εφικτό.

        Αν θέλουμε να λύσουμε πρόβλημα γενικευμένων ιδιοτιμών γίνεται με την
        παρακάτω συνάρτηση.
        \begin{otherlanguage}{english}
            \begin{center}
                [lopt, xopt] = gevp(lmisys, nlfc, options, linit, xinit, target)
            \end{center}
        \end{otherlanguage}
        Εδώ υποχρεωτικό όρισμα είναι και ο όρος \mono{nlfc} που δηλώνει το
        γραμμικό-κλασματικό περιορισμό. Η λύση προκύπτει πάλι με τη συνάρτηση
        \mono{dec2mat} και η τιμή εξόδου \mono{lopt} είναι το ολικό ελάχιστο.

        Τέλος, αν θέλουμε να ελαχιστοποιήσουμε γραμμική αντικειμενική συνάρτηση
        που υπόκειται σε \tl{LMI} περιορισμούς γίνεται με τη συνάρτηση
        \begin{otherlanguage}{english}
            \begin{center}
                [copt, xopt] = mincx(lmisys, c, options, xinit, target).
            \end{center}
        \end{otherlanguage}
        Ο όρος \mono{c} είναι διάνυσμα που πολλαπλασιάζει το διάνυσμα που
        θέλουμε να ελαχιστοποιήσουμε. Ο όρος \mono{copt} είναι το ολικό ελάχιστο
        της αντικειμενικής συνάρτησης και ο όρος \mono{xopt}, το βέλτιστο
        διάνυσμα, σχετίζεται με τους αντίστοιχους πίνακες \tl{LMI} και πάλι με
        τη συνάρτηση \mono{dec2mat}.
\end{enumerate}

\subsection{Παράδειγμα 1}
Ένα χαρακτηριστικό παράδειγμα \tl{LMI} που διατυπώνεται στο βιβλίο
\cite{press2008solving}, είναι η ανισότητα \tl{Riccati}
\begin{equation*}
    A^TX + XA + XBR^{-1}B^TX + Q < 0,
\end{equation*}
όπου ψάχνουμε τον θετικά ορισμένο πίνακα \(X\). Προφανώς η ανισότητα είναι
μη-γραμμική. Όμως κάνοντας χρήση του συμπληρώματος του \tl{Schur} μπορούμε να
διατυπώσουμε την ανισότητα ως \tl{LMI} καθώς οι πίνακες \(A, B, Q\) είναι
σταθεροί. Έτσι έχουμε
\begin{equation}\label{eq:lmi_ric}
    \begin{pmatrix}
        A^TX + XA + Q & XB \\
        B^TX & -R
    \end{pmatrix} < 0, \quad \text{\gr{και }} X > 0.
\end{equation}
Χάριν του παραδείγματος έστω
\begin{equation*}
    A = \begin{pmatrix}
        -2 & -2 & -1 \\
        -3 & -1 & -1 \\
        1 & 0 & -4
    \end{pmatrix},\quad
    B = \begin{pmatrix}
        -1 & 0 \\
        0 & -1 \\
        -1 & -1
    \end{pmatrix},\quad
    Q = \begin{pmatrix}
        -2 & 1 & -2 \\
        1 & -2 & -4 \\
        -2 & -4 & -2
    \end{pmatrix},\quad R = I_2.
\end{equation*}
Παρακάτω παρατίθεται ο κώδικας στο \tl{MATLAB} για το πρόβλημα.
\begin{otherlanguage}{english}
    \lstinputlisting[language=Matlab]{src/lmi1.m}
\end{otherlanguage}
Όπως φαίνεται στον κώδικα στη γραμμή \(8\), ο αριστερός όρος της σχέσης \eqref{eq:lmi_ric} είναι
το \tl{LMI} \(1\) και ο δεξιός το \tl{LMI} \(2\), γραμμή \(13\). Ο πίνακας \(X\)
είναι \(3 \times 3\) συμμετρικός. Τελικά, βρίσκουμε ότι \(t_{min} = -0.3962\) και
ο πίνακας εφικτής λύσης που αντιστοιχεί στη βέλτιστη τιμή επιστρέφεται από τη συνάρτηση
\mono{dec2mat} και είναι
\begin{equation*}
    X =
    \begin{pmatrix}
        1.0329 & 0.4647 & -0.2358 \\
        0.4647 &   0.7790 &  -0.0507 \\
        -0.2358 &  -0.0507 & 1.4336
    \end{pmatrix}.
\end{equation*}

\subsection{Παράδειγμα 2}
Το επόμενο παράδειγμα προέρχεται από το \tl{LMI Control Toolbox} του \tl{MATLAB}
\cite{gahinet1994lmi}. Έστω το πρόβλημα ελαχιστοποίησης
\begin{equation*}
    \begin{aligned}
        & \text{\tl{minimize}} & & \text{\tl{trace(X)}} \\
        & \text{\tl{subject to}} & & A^TX + XA + XBB^T + Q < 0,
    \end{aligned}
\end{equation*}
με δεδομένα
\begin{equation*}
    A = \begin{pmatrix}
        -1 & -2 & 1 \\
        3 & 2 & 1 \\
        1 & -2 & -1
    \end{pmatrix},\quad
    B = \begin{pmatrix}
        1 \\
        0 \\
        1
    \end{pmatrix},\quad
    Q = \begin{pmatrix}
        1 & -1 & 0 \\
        -1 & -3 & -12 \\
        0 & -12 & -36
    \end{pmatrix}.
\end{equation*}
Αποδεικνύεται ότι η βέλτιστη λύση \(X^*\) είναι η λύση της αλγεβρικής εξίσωσης
\tl{Riccati}
\begin{equation*}
    A^TX + XA + XBB^TX + Q = 0.
\end{equation*}
Η λύση της παραπάνω μπορεί να υπολογιστεί από τη συνάρτηση \mono{care} του
\tl{MATLAB} και να συγκριθεί με το αποτέλεσμα της \tl{LMI} ρουτίνας
\mono{mincx}.

Οι περιορισμοί του προβλήματος μπορούν να μετατραπούν σε \tl{LMI} μορφή και
επειδή το ίχνος είναι γραμμική συνάρτηση, προκύπτει το πρόβλημα ελαχιστοποίησης
γραμμικής αντικειμενικής συνάρτησης
\begin{equation*}
    \begin{aligned}
        & \text{\tl{minimize}} & & \text{\tl{trace(X)}} \\
        & \text{\tl{subject to}} & &
        \begin{pmatrix}
            A^TX + XA + Q & XB \\
            B^TX & -I
        \end{pmatrix} < 0.
    \end{aligned}
\end{equation*}
Στις γραμμές \(6\) με \(12\) δηλώνουμε το \tl{LMI} όπως το προηγούμενο παράδειγμα. Στην
γραμμή \(13\) δηλώνουμε την αντικειμενική συνάρτηση, δηλαδή το ίχνος του
\(X\). Άρα το διάνυσμα \(c\) είναι το \(I_3\) καθώς το \tl{MATLAB} αναμένει την
αντικειμενική στη μορφή \(c^Tx\). Με την εντολή \mono{mat2dev} μετατρέπουμε τις
μεταβλητές πινάκων σε μεταβλητές απόφασης για το γραμμικό πρόβλημα. Από τα
\mono{options} θέτουμε το σχετικό σφάλμα \(10^{-5}\). Λύνοντας το
\tl{LMI} πρόβλημα ελαχιστοποίησης βρίσκουμε ότι \( \text{\tl{trace}}(X) =
-18.7166\) και ικανοποιείται με σχετική ακρίβεια \(9.5\cdot10^{-6}\). Η μεταβλητή
πίνακα που αντιστοιχεί στη βέλτιστη τιμή επιστρέφεται από τη συνάρτηση
\mono{dec2mat} και είναι
\begin{equation*}
    X =
    \begin{pmatrix}
        -6.3542 & -5.8895 & 2.2046 \\
        -5.8895 & -6.2855 & 2.2201 \\
        2.2046 & 2.2201 & -6.0771
    \end{pmatrix}.
\end{equation*}
Το αποτέλεσμα μπορεί να συγκριθεί με τη λύση της εξίσωσης \tl{Riccati} που
υπολογίζεται από τη συνάρτηση \mono{care}. Τελικά, βρίσκουμε ότι η απόκλιση της
βέλτιστης τιμής, όπως υπολογίστηκε από την προσέγγιση με τα \tl{LMI}, με τη τιμή
της εξίσωσης \tl{Riccati}, όπως υπολογίστηκε από τη συνάρτηση \mono{care},
γραμμή 17, είναι \(6.539\cdot10^{-5}\).

Παρακάτω παρατίθεται ο κώδικας στο \tl{MATLAB} για το πρόβλημα.
\begin{otherlanguage}{english}
    \lstinputlisting[language=Matlab]{src/lmi2.m}
\end{otherlanguage}
