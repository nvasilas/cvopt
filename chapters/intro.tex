\chapter{Εισαγωγή}
Επιστήμονες από όλους τους κλάδους όπως μηχανικοί, μαθηματικοί, οικονομολόγοι
και άλλοι, έρχονται αντιμέτωποι με προβλήματα που πρέπει να λύσουν. Τα
προβλήματα πολλές φορές απαιτούν ένα βέλτιστο σχεδιασμό, όπως κατανομή σπάνιων πόρων,
σχεδιασμό βιομηχανικών δραστηριοτήτων, ή την εύρεση της τροχιάς ενός πυραύλου.
Στο παρελθόν, ένα μεγάλο εύρος λύσεων θεωρούταν αποδεκτό. Στον μηχανολογικό
σχεδιασμό για παράδειγμα, ήταν συνηθισμένη η χρήση μεγάλων παραγόντων ασφαλείας.
Όμως, λόγω του συνεχούς αυξανόμενου ανταγωνισμού, δεν είναι πλέον αρκετός
ο σχεδιασμός απλά αποδεκτών λύσεων. Σε άλλες περιπτώσεις, όπως για παράδειγμα ο
σχεδιασμός ενός διαστημικού οχήματος, οι αποδεκτές λύσεις είναι περιορισμένες.
Έτσι, γεννήθηκε η ανάγκη για απάντηση στα ερωτήματα όπως, αν κάνουμε αποδοτική
χρήση των σπάνιων πόρων, ή αν μπορούμε να κάνουμε ένα οικονομικότερο σχεδιασμό.
Ως απάντηση στο όλο και αυξανόμενο πεδίο τέτοιων ερωτήσεων, υπήρχε μία πολύ
γρήγορη ανάπτυξη μοντέλων και τεχνικών βελτιστοποίησης. Επίσης, η παράλληλη
ραγδαία εξέλιξη των υπολογιστικών δυνατοτήτων βοήθησε σημαντικά στη χρήση των
τεχνικών που αναπτύχθηκαν.

Η ιδέα της βελτιστοποίησης είναι βαθιά ριζωμένη ως η βασική αρχή στην ανάλυση
πολλών πολύπλοκων αποφάσεων. Χρησιμοποιώντας την ιδέα της βελτιστοποίησης,
προσεγγίζουμε ένα πολύπλοκο πρόβλημα, που περιλαμβάνει την επιλογή τιμών για
έναν αριθμό συγγενικών μεταβλητών, εστιάζοντας σε μία αντικειμενική συνάρτηση
που σχεδιάστηκε για να ποσοτικοποιήσει την επίδοση και να μετρήσει την ποιότητα
της απόφασης. Αυτή η αντικειμενική μεγιστοποιείται ή ελαχιστοποιείται και
υπόκειται σε περιορισμούς που ενδέχεται να περιορίσουν την επιλογή των
μεταβλητών απόφασης. Αν μία κατάλληλη πτυχή τους προβλήματος μπορεί να
απομονωθεί και να χαρακτηριστεί από μία αντικειμενική, είτε αυτό είναι κέρδος ή
ζημία στον επιχειρηματικό τομέα, είτε ταχύτητα ή απόσταση σε ένα φυσικό πρόβλημα
και τα λοιπά, η βελτιστοποίηση είναι το κατάλληλο εργαλείο για την ανάλυση των
προβλημάτων αυτών.

Είναι φυσικά εξαιρετικά σπάνια η περίπτωση όπου είναι εφικτό να περιγράψουμε
πλήρως όλες τις περίπλοκες αλληλεπιδράσεις των μεταβλητών, των περιορισμών και
να σχεδιάσουμε αντικειμενική συνάρτηση που αντιπροσωπεύει ένα πολύπλοκο
πρόβλημα απόφασης. Έτσι, η διατύπωση ενός προβλήματος βελτιστοποίησης μπορεί να
θεωρηθεί μόνο ως προσέγγιση του πραγματικού προβλήματος. Η ικανότητα στη
μοντελοποίηση, δηλαδή να συμπεριλάβουμε όλα τα σημαντικά στοιχεία ενός προβλήματος, και
η σωστή κρίση στην ερμηνεία των αποτελεσμάτων είναι απαραίτητα για την εξαγωγή
ορθών συμπερασμάτων. Η βελτιστοποίηση συνεπώς, πρέπει να θεωρείται ως ένα
εργαλείο για την καλύτερη κατανόηση και ανάλυση του προβλήματος.

Η ικανότητα και η ορθή κρίση, όσον αφορά τη διατύπωση του προβλήματος και την
ερμηνεία των αποτελεσμάτων, ενισχύεται μέσω της πρακτικής εμπειρίας και τη
μελέτη της θεωρίας. Η διατύπωση του προβλήματος πάντα περιλαμβάνει κάποιο
συμβιβασμό μεταξύ της αύξησης της μαθηματικής πολυπλοκότητας του προβλήματος και
την απόκλιση του μοντέλου από το πραγματικό πρόβλημα. Για να μπορέσει κάποιος να
επιλέξει τη χρυσή τομή πρέπει να είναι σε θέση να αναγνωρίσει την ιδιαιτερότητα
του κάθε προβλήματος, κάτι που είναι εφικτό μόνο με τη βαθιά μελέτη του τομέα.

Στην παρούσα εργασία παρουσιάζουμε κάποιους βασικούς τομείς της βελτιστοποίησης.
Σε κάθε κεφάλαιο αναλύουμε τη θεωρία που διέπει την κάθε κατηγορία προβλημάτων,
αναλύουμε κάποιον αλγόριθμο για την επίλυση του προβλήματος και δίνουμε κάποια
τυπικά παραδείγματα στο \tl{MATLAB}. Η εισαγωγή βασίστηκε ως επί τον πλείστον 
στα βιβλία \cite{bazaraa2013nonlinear} και \cite{luenberger2008linear}, και ο
ενδιαφερόμενος αναγνώστης παραπέμπεται εκεί για περαιτέρω πληροφορίες.
