\chapter{Συμπεράσματα}
Στην παρούσα εργασία αναλύσαμε σε θεωρητικό επίπεδο διάφορες τεχνικές
βελτιστοποίησης. Τα προβλήματα βελτιστοποίησης εμφανίζονται σε κάθε τομέα της
επιστήμης και είναι απαραίτητο να γνωρίζουμε κάποια θεωρητικά κομμάτια των
αλγορίθμων. Ακόμα, το μέγεθος των προβλημάτων που εμφανίζονται καθιστούν
απαραίτητη τη χρήση εξειδικευμένου λογισμικού, όπως το \tl{MATLAB}, για την
επίλυση αυτών.

Στο πρώτο κεφάλαιο δώσαμε κάποια ιστορικά στοιχεία σχετικά με
το γραμμικό προγραμματισμό. Στη συνέχεια πραγματοποιήσαμε μία βασική εισαγωγή
στο θεωρητικό κομμάτι του γραμμικού προγραμματισμού, και τέλος παρουσιάσαμε
κάποια βασικά παραδείγματα στο \tl{MATLAB}. Το κεφάλαιο αυτό είναι από τα πιο
σημαντικά κομμάτια στον τομέα της αριθμητικής βελτιστοποίησης. Καθώς όχι μόνο έχει
μελετηθεί εκτενώς από θεωρητική σκοπιά αλλά έχουν αναπτυχθεί ισχυροί
αλγόριθμοι που μας επιτρέπουν τη γρήγορη και αποτελεσματική επίλυση των
προβλημάτων. Επίσης, τα γραμμικά προβλήματα βρίσκονται στην καρδία κάθε
προβλήματος και πάντα επιδιώκουμε να εκφράσουμε το πρόβλημα με γραμμική μορφή,
αν φυσικά η φύση του προβλήματος το επιτρέπει.

Βασικό μειονέκτημα του γραμμικού προγραμματισμού αποτελεί το γεγονός ότι πολλές
το πρόβλημα δε μπορεί να περιγραφεί ικανοποιητικά από γραμμικούς όρους. Για αυτό
έχουν αναπτυχθεί άλλες τεχνικές βελτιστοποίησης. Στο δεύτερο κεφάλαιο αναφέραμε
κάποια θεωρητικά στοιχεία του τετραγωνικού προγραμματισμού, δηλαδή όταν η
συνάρτηση που θέλουμε να βελτιστοποιήσουμε είναι τετραγωνικής μορφής. Αναλύσαμε
τη μέθοδο ενεργών περιορισμών που είναι ένας από τους επικρατέστερους
αλγορίθμους για την επίλυση προβλημάτων τετραγωνικού προγραμματισμού και
παραθέσαμε κάποια παραδείγματα επίλυσης τέτοιων προβλημάτων.

Στο τρίτο κεφάλαιο παραθέτουμε τα απαραίτητα θεωρητικά κομμάτια της κυρτής
θεωρίας. Είναι ένας κλάδος που έχει αναπτυχθεί ευρέως και παρουσιάζει μεγάλο
ενδιαφέρον καθώς έχουν αναπτυχθεί αποτελεσματικοί μέθοδοι επίλυσης προβλημάτων
κυρτής βελτιστοποίησης. Δεν υπάρχει αναλυτική λύση για τα προβλήματα αυτά, όμως έχει
αποδειχθεί ότι σε κάποιες περιπτώσεις δύναται να λύσουν προβλήματα με εξαιρετική
ακρίβεια και με πολυπλοκότητα που δε ξεπερνά την πολυωνυμική σε σχέση με το μέγεθος
του προβλήματος. Η έρευνα σε αλγορίθμους εσωτερικού σημείου για γενικά
μη-γραμμικά κυρτά προβλήματα συνεχίζεται και είναι πολύ ενεργή καθώς τα
αποτελέσματα είναι πολύ καλά. Στο τέταρτο κεφάλαιο παρουσιάζουμε τη μέθοδο
φράγματος, ενός αλγορίθμου που υπάγεται στην κατηγορία των μεθόδων εσωτερικών
σημείων.

Στο τελευταίο κεφάλαιο κάνουμε μία μικρή αναφορά στις γραμμικές ανισότητες
πινάκων. Είναι πολύ συνηθισμένο στον έλεγχο οι περιορισμοί να είναι μη-γραμμικοί
όμως κάποιες τεχνικές να μετασχηματίζονται σε \tl{LMI} μορφή, δηλαδή σε
γραμμικές ανισότητες πινάκων. Επικεντρωθήκαμε κυρίως στον τρόπο που επιλύονται
τα προβλήματα αυτά στο περιβάλλον προγραμματισμού \tl{MATLAB}.

Βασικό κομμάτι των προβλημάτων βελτιστοποίησης είναι η κατανόηση του προβλήματος
και η ευχέρεια να καταλάβουμε που κατατάσσεται, για παράδειγμα αν πρόκειται για
γραμμικό πρόβλημα ή αν μπορεί να εκφραστεί ικανοποιητικά ως γραμμικό. Το επόμενο
βήμα είναι η μαθηματική διατύπωση αυτού. Φυσικά δεν είναι εύκολο να εκφραστεί
ένα φυσικό πρόβλημα μαθηματικά, είτε λόγω μεγέθους είτε λόγω πολυπλοκότητας του
προβλήματος και ούτω καθεξής. Τέλος, η υλοποίηση προγράμματος που επιλύει το
πρόβλημα δεν είναι εύκολη. Παρόλο αυτά, έχουν αναπτυχθεί διάφορα προγράμματα
ανοιχτού και κλειστού κώδικα που διευκολύνουν τη διαδικασία της επίλυσης.
